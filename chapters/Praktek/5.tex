
\section{Nico Ekklesia Sembiring}
\subsection{Buatlah fungsi untuk mendapatkan data langsung dari arduino}
\lstinputlisting[caption = Mendapatkan data dari Arduino., firstline=8, lastline=14]{src/5/Praktek/1174096/1174096_realtime.py}

\subsection{Buatlah fungsi untuk mendapatkan data langsung dari arduino dengan looping}
\lstinputlisting[caption = Mendapatkan data langsung dari Arduino dengan looping., firstline=8, lastline=15]{src/5/Praktek/1174096/1174096_save.py}

\subsection{Buatlah fungsi untuk mendapatkan data dari arduino dan langsung ditulis kedalam file csv}
\lstinputlisting[caption = Mendapatkan data dari Arduino dan langsung ditulis kedalam file CSV., firstline=16, lastline=30]{src/5/Praktek/1174096/1174096_realtime.py}

\begin{figure}[H]
	\includegraphics[width=9cm]{figures/5/Praktek/1174096/hasilcsv.png}
	\caption{Hasil dari pembacaan data dari arduino dalam bentuk file CSV.}
	\centering
\end{figure}

\subsection{Buatlah fungsi untuk membaca file csv hasil arduino dan mengembalikan ke fungsi}
\lstinputlisting[caption = Membaca file CSV hasil Arduino dan mengembalikan fungsi., firstline=8, lastline=16]{src/5/Praktek/1174096/1174096_csv.py}

\subsection{Pengecekan Plagiarisme Praktek}
\begin{figure}[H]
	\includegraphics[width=9cm]{figures/5/Praktek/1174096/Plagiarismepraktek.png}
	\centering
\end{figure}

\subsection{Ketrampilan Penanganan Error}
\textbf{Tuliskan peringatan error yang didapat dari mengerjakan praktek ketiga ini, dan jelaskan cara penanganan error tersebut. dan Buatlah satu fungsi yang menggunakan gunakan try except untuk menanggulangi error tersebut}

Peringatan error yang saya temui pada praktek Chapter 5 ini, adalah:
\begin{itemize}
	\item Name Error
	NameError adalah exception yang terjadi ketika kode melakukan eksekusi terhadap local name atau global name yang tidak terdefinisi oleh perangkat. Solusi yang dapat dilakukan adalah dengan memastikan variabel atau fungsi yang dipanggil ada atau tidak salah ketik.
	
	\item Syntax Errors
	Syntax Errors adalah suatu keadaan saat  terjadi kesalahan penulisan pada kode python. Cara memperbaikinya adalah dengan memperbaiki penulisan kode yang salah.
	
	\item Type Error
	TypeError adalah exception yang terjadi pada saat dilakukannya eksekusi terhadap suatu operasi atau fungsi dengan type object yang tidak sesuai. Cara yang dilakukan untuk mengatasinya error ini adalah mengkoversi varibelnya sesuai dengan tipe data yang akan digunakan.
\end{itemize}

\textbf{Penanggulangan Error menggunakan Try Except}
\lstinputlisting[caption = Penanggulangan error menggunakan Try Except., firstline=8, lastline=23]{src/5/Praktek/1174096/1174096.py}

\subsection{Pengecekan Plagiarisme Penanganan Error}
\begin{figure}[H]
	\includegraphics[width=9cm]{figures/5/Praktek/1174096/Plagiarismeerror.png}
	\centering
\end{figure}

\section{Rahmatul Ridha}
\subsection{Soal 1}
Isi jawaban soal ke-1

Kalau mau dibikin paragrap \textbf{cukup enter aja}, tidak usah pakai \verb|par| dsb

%\subsection{Soal 2}
%Isi jawaban soal ke-2

%\subsection{Soal 3}
%Isi jawaban soal ke-3

\section{Tomy Prawoto}
\subsection{Soal 1}
Isi jawaban soal ke-1

Kalau mau dibikin paragrap \textbf{cukup enter aja}, tidak usah pakai \verb|par| dsb

%\subsection{Soal 2}
%Isi jawaban soal ke-2

%\subsection{Soal 3}
%Isi jawaban soal ke-3
